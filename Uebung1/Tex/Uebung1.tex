\documentclass[11pt,titlepage]{article}

%Laenderspezifische Einstellungen bzgl. Rechtschreibung, Sonderzeichen und Kodierung
\usepackage[utf8]{inputenc}
\usepackage[english]{babel}
\usepackage[T1]{fontenc}
\usepackage{titlesec}
\usepackage{graphicx}

\usepackage{listings}
\usepackage{color}
\usepackage{courier}
\definecolor{light-gray}{gray}{0.85}
\lstset{
language=C++,
numbers=left,
breaklines=true,
backgroundcolor=\color{light-gray},
tabsize=2,
basicstyle=\footnotesize\ttfamily,
frame=single,
inputencoding=utf8,
extendedchars=true,
showstringspaces=false,
literate =
	{ä}{{\"a}}1
	{ö}{{\"o}}1
	{ü}{{\"u}}1
	{Ä}{{\"A}}1
	{Ö}{{\"O}}1
	{Ü}{{\"U}}1
	{ß}{{\ss}}1
	{ₙ}{{$_n$}}1
}

\def\ContinueLineNumber{\lstset{firstnumber=last}}

\usepackage[
	a4paper,
	top = 2cm,
	bottom = 2 cm,
	left = 2cm,
	right = 2cm,
	headheight = 15pt,
	includeheadfoot
	]{geometry}
\usepackage{fancyhdr}
\usepackage{amssymb}
\usepackage{amsmath}
\usepackage[english]{varioref}
\usepackage{hyperref}

\fancypagestyle{firstPage}{
	\fancyhead[LE,RO]{\textsc{Page} \thepage}
	\fancyhead[LO,RE]{}
	\fancyheadoffset[LE,RO]{2.5cm}
	\renewcommand{\headrulewidth}{0.5pt}
	
	\fancyfoot[LE,RO]{\tiny{Assignment 1, Felix Dreßler (k12105003), created: \today}}
	\cfoot{}
	\fancyfootoffset[Le,RO]{2.5cm}
	\renewcommand{\footrulewidth}{0.5pt}
}

\fancypagestyle{fancy}{
	\fancyhead[R]{Page \thepage}
	\fancyhead[L]{\leftmark}
	\renewcommand{\headrulewidth}{1.25pt}

	\fancyfoot[L]{\tiny{Programming 2 - Assignment 1, created: \today}}
	\fancyfoot[R]{\tiny{ Felix Dreßler (k12105003)}}
	\cfoot{}
	\renewcommand{\footrulewidth}{1.25pt}
}

\setlength{\headsep}{10mm}
\setlength{\footskip}{10mm}

\setlength{\parindent}{0mm}
\setlength{\parskip}{1.1ex plus0.25ex minus0.25ex}
\setlength{\tabcolsep}{0.2cm} % for the horizontal padding

\pagestyle{fancy}

\title{Programming 2 - Assignment 1}
\author{Felix Dreßler (k12105003)}
\date{\today} %Erstellungsdatum

\begin{document}
\maketitle
	\section{Testing the Program}
		For testing purposes a series of tests was performed. The program was tested first without the implementation of collisions between different atoms for an fixed input saved in "Input.txt" and random generated atoms.
		Next the program was tested including the implementation of collision between different atoms. This was executed once using the same fixed "Input.txt" and with random generated atoms.
		
		For each test, the text output of the program was recorded, stating the initial values of the atoms. Furthermore, screenshots of the initial-state and the end-state are included.
		\subsection{Tests without atom-collison}
		
		\subsection{Tests with atom-collsion}
		
	\section{The Program - Main.cpp}	
		\subsection{Header of Main.cpp}
			The following paragraph shows the beginning of the File "Main.cpp" of the "Atoms" Project. 
			We see, that in comparison to the given Header in the assignment, there is an "Auxiliary.h" included in line 32. This Header-File will be discussed in the next section.
			
			There are also four global variables defined in lines 34 to 38. W and H are width and height of the created window, in which the atoms will be simulated.
			S describes the time that will pass between each frame. It will be passed to the Sleep function that is describes in lines 14 to 25.
			\lstinputlisting[firstline=0,lastline=38]{Main.cpp}
			
			
		\subsection{structure "Atom"}	
			\ContinueLineNumber
			\lstinputlisting[firstline=40,lastline=61]{Main.cpp}
			
		\subsection{Function "number"}
			\ContinueLineNumber
			\lstinputlisting[firstline=63,lastline=101]{Main.cpp}
			
		\subsection{Function "init"}
			\ContinueLineNumber
			\lstinputlisting[firstline=103,lastline=209]{Main.cpp}
			
		\subsection{Function "Draw"}
			\ContinueLineNumber
			\lstinputlisting[firstline=211,lastline=231]{Main.cpp}
			
		\subsection{Function "Update"}
			\ContinueLineNumber
			\lstinputlisting[firstline=233,lastline=332]{Main.cpp}
			
		\subsection{Main}
			\ContinueLineNumber
			\lstinputlisting[firstline=334,lastline=354]{Main.cpp}
			
		\section{The Program - Auxiliary}
		For better clarity, auxiliary functions were outsourced to the files 
		"Auxiliary.h" and "Auxiliary.cpp".
			\subsection{Auciliary.h}
			In the file "Auxiliary.h", the auxiliary functions are declared.
				\lstinputlisting{Auxiliary.h}
			\subsection{Auxiliary.cpp}
			In the file "Auxiliary.cpp" the auxiliary functions are defined.
			
			If the random Function (staring in line 407) receives a wide range in which a random number should be generated, it struggles to give out all possible values.
			For example when creating a random color in the int range 0 to 0xFFFFFF we will never see "red" as output of "random". This could be resolved by creating three independent random numbers (RGB) and later combining them.
				\lstinputlisting{Auxiliary.cpp}
\end{document}