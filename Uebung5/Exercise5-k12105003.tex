\documentclass[11pt,titlepage]{article}

%Laenderspezifische Einstellungen bzgl. Rechtschreibung, Sonderzeichen und Kodierung
\usepackage[utf8]{inputenc}
\usepackage[english]{babel}
\usepackage[T1]{fontenc}
\usepackage{titlesec}
\usepackage{graphicx}
%\usepackage{subcaption}

\usepackage{listings}
\usepackage{color}
\usepackage{courier}
\definecolor{light-gray}{gray}{0.85}
\definecolor{dark-green}{rgb}{0.05,0.65,0.3}
\lstset{
language=C++,
numbers=left,
breaklines=true,
backgroundcolor=\color{light-gray},
tabsize=2,
basicstyle=\footnotesize\ttfamily,
frame=single,
inputencoding=utf8,
extendedchars=true,
showstringspaces=false,
commentstyle=\color{dark-green}\ttfamily,
literate =
	{ä}{{\"a}}1
	{ö}{{\"o}}1
	{ü}{{\"u}}1
	{Ä}{{\"A}}1
	{Ö}{{\"O}}1
	{Ü}{{\"U}}1
	{ß}{{\ss}}1
	{ₙ}{{$_n$}}1
}

\def\ContinueLineNumber{\lstset{firstnumber=last}}
\def\StartLineAt#1{\lstset{firstnumber=#1}}

\usepackage[
	a4paper,
	top = 2cm,
	bottom = 2 cm,
	left = 2cm,
	right = 2cm,
	headheight = 15pt,
	includeheadfoot
	]{geometry}
\usepackage{fancyhdr}
\usepackage{amssymb}
\usepackage{amsmath}
\usepackage[english]{varioref}
\usepackage{hyperref}

\fancypagestyle{fancy}{
	\fancyhead[R]{Page \thepage}
	\fancyhead[L]{\leftmark}
	\renewcommand{\headrulewidth}{1.25pt}

	\fancyfoot[L]{\tiny{Programming 2 - Assignment 5, created: \today}}
	\fancyfoot[R]{\tiny{ Felix Dreßler (k12105003)}}
	\cfoot{}
	\renewcommand{\footrulewidth}{1.25pt}
}

\setlength{\headsep}{10mm}
\setlength{\footskip}{10mm}

\setlength{\parindent}{0mm}
\setlength{\parskip}{1.1ex plus0.25ex minus0.25ex}
\setlength{\tabcolsep}{0.2cm} % for the horizontal padding

\pagestyle{fancy}

\title{Programming 2 - Assignment 5}
\author{Felix Dreßler (k12105003)\\ email \href{mailto:FelixDressler01@gmail.com}{FelixDressler01@gmail.com}}
\date{\today} %Erstellungsdatum

\begin{document}
\maketitle
	
	\section{Testing the Program}
		\subsection{Main.cpp}
		For testing the \emph{Main.cpp} was adapted to work with the new templates.
			\lstinputlisting[]{Assignment5/Assignment5/Main.cpp}
		\subsection{output}
		The output was the same as in the previous project.
			\begin{lstlisting}
				(-5+2*x^1+-3*x^3)
				(-10+4*x^1+-6*x^3)
				(50+-40*x^1+8*x^2+60*x^3+-24*x^4+18*x^6)
				0
				(-5+2*x^1+-3*x^3)
				0
				(5+-2*x^1+3*x^3)
				0
				((-5+2*x^1+-3*x^3)+(-10+4*x^1+-6*x^3)*y^1)
				((-5+2*x^1+-3*x^3))
				((-10+4*x^1+-6*x^3)+(-10+4*x^1+-6*x^3)*y^1)
				((25+-20*x^1+4*x^2+30*x^3+-12*x^4+9*x^6)+(50+-40*x^1+8*x^2+60*x^3+-24*x^4+18*x^6)*y^1)
			\end{lstlisting}
	
\newpage	
	\section{RecPoly.h}
	Because this project was about using the previous code and adapt it to work with templates, the memory leak in line 186-189 still exists.
	
			\lstinputlisting[]{Assignment5/Assignment5/RecPoly.h}
	
\newpage	
	\section{Integer}
		\subsection{Integer.h}
			\lstinputlisting[]{Assignment5/Assignment5/Integer.h}
		\subsection{Integer.cpp}
			\lstinputlisting[]{Assignment5/Assignment5/Integer.cpp}
	
			
\end{document}